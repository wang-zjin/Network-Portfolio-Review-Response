% Options for packages loaded elsewhere
\PassOptionsToPackage{unicode}{hyperref}
\PassOptionsToPackage{hyphens}{url}
%
\documentclass[
]{article}
\usepackage{amsmath,amssymb}
\usepackage{iftex}
\ifPDFTeX
  \usepackage[T1]{fontenc}
  \usepackage[utf8]{inputenc}
  \usepackage{textcomp} % provide euro and other symbols
\else % if luatex or xetex
  \usepackage{unicode-math} % this also loads fontspec
  \defaultfontfeatures{Scale=MatchLowercase}
  \defaultfontfeatures[\rmfamily]{Ligatures=TeX,Scale=1}
\fi
\usepackage{lmodern}
\ifPDFTeX\else
  % xetex/luatex font selection
\fi
% Use upquote if available, for straight quotes in verbatim environments
\IfFileExists{upquote.sty}{\usepackage{upquote}}{}
\IfFileExists{microtype.sty}{% use microtype if available
  \usepackage[]{microtype}
  \UseMicrotypeSet[protrusion]{basicmath} % disable protrusion for tt fonts
}{}
\makeatletter
\@ifundefined{KOMAClassName}{% if non-KOMA class
  \IfFileExists{parskip.sty}{%
    \usepackage{parskip}
  }{% else
    \setlength{\parindent}{0pt}
    \setlength{\parskip}{6pt plus 2pt minus 1pt}}
}{% if KOMA class
  \KOMAoptions{parskip=half}}
\makeatother
\usepackage{xcolor}
\usepackage[margin=1in]{geometry}
\usepackage{color}
\usepackage{fancyvrb}
\newcommand{\VerbBar}{|}
\newcommand{\VERB}{\Verb[commandchars=\\\{\}]}
\DefineVerbatimEnvironment{Highlighting}{Verbatim}{commandchars=\\\{\}}
% Add ',fontsize=\small' for more characters per line
\usepackage{framed}
\definecolor{shadecolor}{RGB}{248,248,248}
\newenvironment{Shaded}{\begin{snugshade}}{\end{snugshade}}
\newcommand{\AlertTok}[1]{\textcolor[rgb]{0.94,0.16,0.16}{#1}}
\newcommand{\AnnotationTok}[1]{\textcolor[rgb]{0.56,0.35,0.01}{\textbf{\textit{#1}}}}
\newcommand{\AttributeTok}[1]{\textcolor[rgb]{0.13,0.29,0.53}{#1}}
\newcommand{\BaseNTok}[1]{\textcolor[rgb]{0.00,0.00,0.81}{#1}}
\newcommand{\BuiltInTok}[1]{#1}
\newcommand{\CharTok}[1]{\textcolor[rgb]{0.31,0.60,0.02}{#1}}
\newcommand{\CommentTok}[1]{\textcolor[rgb]{0.56,0.35,0.01}{\textit{#1}}}
\newcommand{\CommentVarTok}[1]{\textcolor[rgb]{0.56,0.35,0.01}{\textbf{\textit{#1}}}}
\newcommand{\ConstantTok}[1]{\textcolor[rgb]{0.56,0.35,0.01}{#1}}
\newcommand{\ControlFlowTok}[1]{\textcolor[rgb]{0.13,0.29,0.53}{\textbf{#1}}}
\newcommand{\DataTypeTok}[1]{\textcolor[rgb]{0.13,0.29,0.53}{#1}}
\newcommand{\DecValTok}[1]{\textcolor[rgb]{0.00,0.00,0.81}{#1}}
\newcommand{\DocumentationTok}[1]{\textcolor[rgb]{0.56,0.35,0.01}{\textbf{\textit{#1}}}}
\newcommand{\ErrorTok}[1]{\textcolor[rgb]{0.64,0.00,0.00}{\textbf{#1}}}
\newcommand{\ExtensionTok}[1]{#1}
\newcommand{\FloatTok}[1]{\textcolor[rgb]{0.00,0.00,0.81}{#1}}
\newcommand{\FunctionTok}[1]{\textcolor[rgb]{0.13,0.29,0.53}{\textbf{#1}}}
\newcommand{\ImportTok}[1]{#1}
\newcommand{\InformationTok}[1]{\textcolor[rgb]{0.56,0.35,0.01}{\textbf{\textit{#1}}}}
\newcommand{\KeywordTok}[1]{\textcolor[rgb]{0.13,0.29,0.53}{\textbf{#1}}}
\newcommand{\NormalTok}[1]{#1}
\newcommand{\OperatorTok}[1]{\textcolor[rgb]{0.81,0.36,0.00}{\textbf{#1}}}
\newcommand{\OtherTok}[1]{\textcolor[rgb]{0.56,0.35,0.01}{#1}}
\newcommand{\PreprocessorTok}[1]{\textcolor[rgb]{0.56,0.35,0.01}{\textit{#1}}}
\newcommand{\RegionMarkerTok}[1]{#1}
\newcommand{\SpecialCharTok}[1]{\textcolor[rgb]{0.81,0.36,0.00}{\textbf{#1}}}
\newcommand{\SpecialStringTok}[1]{\textcolor[rgb]{0.31,0.60,0.02}{#1}}
\newcommand{\StringTok}[1]{\textcolor[rgb]{0.31,0.60,0.02}{#1}}
\newcommand{\VariableTok}[1]{\textcolor[rgb]{0.00,0.00,0.00}{#1}}
\newcommand{\VerbatimStringTok}[1]{\textcolor[rgb]{0.31,0.60,0.02}{#1}}
\newcommand{\WarningTok}[1]{\textcolor[rgb]{0.56,0.35,0.01}{\textbf{\textit{#1}}}}
\usepackage{graphicx}
\makeatletter
\def\maxwidth{\ifdim\Gin@nat@width>\linewidth\linewidth\else\Gin@nat@width\fi}
\def\maxheight{\ifdim\Gin@nat@height>\textheight\textheight\else\Gin@nat@height\fi}
\makeatother
% Scale images if necessary, so that they will not overflow the page
% margins by default, and it is still possible to overwrite the defaults
% using explicit options in \includegraphics[width, height, ...]{}
\setkeys{Gin}{width=\maxwidth,height=\maxheight,keepaspectratio}
% Set default figure placement to htbp
\makeatletter
\def\fps@figure{htbp}
\makeatother
\setlength{\emergencystretch}{3em} % prevent overfull lines
\providecommand{\tightlist}{%
  \setlength{\itemsep}{0pt}\setlength{\parskip}{0pt}}
\setcounter{secnumdepth}{-\maxdimen} % remove section numbering
\ifLuaTeX
  \usepackage{selnolig}  % disable illegal ligatures
\fi
\usepackage{bookmark}
\IfFileExists{xurl.sty}{\usepackage{xurl}}{} % add URL line breaks if available
\urlstyle{same}
\hypersetup{
  pdftitle={R Notebook},
  hidelinks,
  pdfcreator={LaTeX via pandoc}}

\title{R Notebook}
\author{}
\date{\vspace{-2.5em}}

\begin{document}
\maketitle

This is an \href{http://rmarkdown.rstudio.com}{R Markdown} Notebook.
When you execute code within the notebook, the results appear beneath
the code.

Try executing this chunk by clicking the \emph{Run} button within the
chunk or by placing your cursor inside it and pressing
\emph{Cmd+Shift+Enter}.

\begin{Shaded}
\begin{Highlighting}[]
\FunctionTok{library}\NormalTok{(MASS) }\CommentTok{\# For matrix operations}

\CommentTok{\# Function to generate a random correlation matrix}
\NormalTok{generate\_correlation\_matrix }\OtherTok{\textless{}{-}} \ControlFlowTok{function}\NormalTok{(n) \{}
  \CommentTok{\# Step 1: Generate a random matrix with values between {-}1 and 1}
\NormalTok{  A }\OtherTok{\textless{}{-}} \FunctionTok{matrix}\NormalTok{(}\FunctionTok{runif}\NormalTok{(n }\SpecialCharTok{*}\NormalTok{ n, }\AttributeTok{min =} \SpecialCharTok{{-}}\DecValTok{1}\NormalTok{, }\AttributeTok{max =} \DecValTok{1}\NormalTok{), }\AttributeTok{nrow =}\NormalTok{ n, }\AttributeTok{ncol =}\NormalTok{ n)}
  
  \CommentTok{\# Step 2: Symmetrize the matrix}
\NormalTok{  A }\OtherTok{\textless{}{-}}\NormalTok{ (A }\SpecialCharTok{+} \FunctionTok{t}\NormalTok{(A)) }\SpecialCharTok{/} \DecValTok{2}
  
  \CommentTok{\# Step 3: Make the matrix positive definite}
  \CommentTok{\# Using \textasciigrave{}nearPD\textasciigrave{} function from Matrix package for numerical stability}
  \FunctionTok{library}\NormalTok{(Matrix)}
\NormalTok{  correlation\_matrix }\OtherTok{\textless{}{-}} \FunctionTok{as.matrix}\NormalTok{(}\FunctionTok{nearPD}\NormalTok{(A, }\AttributeTok{corr =} \ConstantTok{TRUE}\NormalTok{)}\SpecialCharTok{$}\NormalTok{mat)}
  
  \FunctionTok{return}\NormalTok{(correlation\_matrix)}
\NormalTok{\}}

\NormalTok{n }\OtherTok{\textless{}{-}} \DecValTok{4}  \CommentTok{\# Number of rows and columns}
\CommentTok{\#set.seed(123)  \# Set seed for reproducibility}
\NormalTok{cor\_matrix }\OtherTok{\textless{}{-}} \FunctionTok{generate\_correlation\_matrix}\NormalTok{(n)}
\CommentTok{\#A\_matrix = cor\_matrix}
\NormalTok{A\_matrix }\OtherTok{=}\NormalTok{ cor\_matrix }\SpecialCharTok{{-}} \FunctionTok{diag}\NormalTok{(}\DecValTok{1}\NormalTok{,n,n)}
\FunctionTok{print}\NormalTok{(A\_matrix)}
\end{Highlighting}
\end{Shaded}

\begin{verbatim}
##             [,1]      [,2]       [,3]        [,4]
## [1,]  0.00000000 0.2331854 -0.1237541 -0.03486696
## [2,]  0.23318537 0.0000000  0.3125351  0.10442570
## [3,] -0.12375411 0.3125351  0.0000000 -0.20502700
## [4,] -0.03486696 0.1044257 -0.2050270  0.00000000
\end{verbatim}

\begin{Shaded}
\begin{Highlighting}[]
\CommentTok{\# Generate random values for the diagonal (strictly between 0 and 1)}
\NormalTok{diagonal\_values }\OtherTok{\textless{}{-}} \FunctionTok{runif}\NormalTok{(n, }\AttributeTok{min =} \DecValTok{0}\NormalTok{, }\AttributeTok{max =} \DecValTok{1}\NormalTok{)}

\CommentTok{\# Create the diagonal matrix}
\NormalTok{Gamma\_matirx }\OtherTok{\textless{}{-}} \FunctionTok{diag}\NormalTok{(diagonal\_values)}

\CommentTok{\# Print the matrix}
\FunctionTok{print}\NormalTok{(Gamma\_matirx)}
\end{Highlighting}
\end{Shaded}

\begin{verbatim}
##           [,1]      [,2]      [,3]      [,4]
## [1,] 0.8068436 0.0000000 0.0000000 0.0000000
## [2,] 0.0000000 0.0226871 0.0000000 0.0000000
## [3,] 0.0000000 0.0000000 0.9750105 0.0000000
## [4,] 0.0000000 0.0000000 0.0000000 0.3029493
\end{verbatim}

\begin{Shaded}
\begin{Highlighting}[]
\NormalTok{A\_matrix}\SpecialCharTok{\%*\%}\NormalTok{Gamma\_matirx}
\end{Highlighting}
\end{Shaded}

\begin{verbatim}
##             [,1]        [,2]       [,3]        [,4]
## [1,]  0.00000000 0.005290300 -0.1206616 -0.01056292
## [2,]  0.18814412 0.000000000  0.3047250  0.03163569
## [3,] -0.09985021 0.007090515  0.0000000 -0.06211279
## [4,] -0.02813218 0.002369116 -0.1999035  0.00000000
\end{verbatim}

\begin{Shaded}
\begin{Highlighting}[]
\NormalTok{Gamma\_matirx}\SpecialCharTok{\%*\%}\NormalTok{A\_matrix}
\end{Highlighting}
\end{Shaded}

\begin{verbatim}
##             [,1]       [,2]         [,3]         [,4]
## [1,]  0.00000000 0.18814412 -0.099850212 -0.028132185
## [2,]  0.00529030 0.00000000  0.007090515  0.002369116
## [3,] -0.12066155 0.30472498  0.000000000 -0.199903471
## [4,] -0.01056292 0.03163569 -0.062112790  0.000000000
\end{verbatim}

\begin{Shaded}
\begin{Highlighting}[]
\FunctionTok{library}\NormalTok{(expm)}
\end{Highlighting}
\end{Shaded}

\begin{verbatim}
## Warning: package 'expm' was built under R version 4.4.1
\end{verbatim}

\begin{verbatim}
## 
## Attaching package: 'expm'
\end{verbatim}

\begin{verbatim}
## The following object is masked from 'package:Matrix':
## 
##     expm
\end{verbatim}

\begin{Shaded}
\begin{Highlighting}[]
\NormalTok{i}\OtherTok{=}\DecValTok{9}
\NormalTok{(Gamma\_matirx}\SpecialCharTok{\%\^{}\%}\NormalTok{ i)}\SpecialCharTok{*}\NormalTok{(A\_matrix}\SpecialCharTok{\%\^{}\%}\NormalTok{ i)}
\end{Highlighting}
\end{Shaded}

\begin{verbatim}
##               [,1]          [,2]          [,3]          [,4]
## [1,] -9.039468e-05  0.000000e+00  0.0000000000  0.000000e+00
## [2,]  0.000000e+00 -1.969355e-18  0.0000000000  0.000000e+00
## [3,]  0.000000e+00  0.000000e+00 -0.0009531712  0.000000e+00
## [4,]  0.000000e+00  0.000000e+00  0.0000000000 -1.035021e-08
\end{verbatim}

\begin{Shaded}
\begin{Highlighting}[]
\NormalTok{(A\_matrix}\SpecialCharTok{\%\^{}\%}\NormalTok{i)}\SpecialCharTok{*}\NormalTok{(Gamma\_matirx}\SpecialCharTok{\%\^{}\%}\NormalTok{i)}
\end{Highlighting}
\end{Shaded}

\begin{verbatim}
##               [,1]          [,2]          [,3]          [,4]
## [1,] -9.039468e-05  0.000000e+00  0.0000000000  0.000000e+00
## [2,]  0.000000e+00 -1.969355e-18  0.0000000000  0.000000e+00
## [3,]  0.000000e+00  0.000000e+00 -0.0009531712  0.000000e+00
## [4,]  0.000000e+00  0.000000e+00  0.0000000000 -1.035021e-08
\end{verbatim}

Add a new chunk by clicking the \emph{Insert Chunk} button on the
toolbar or by pressing \emph{Cmd+Option+I}.

When you save the notebook, an HTML file containing the code and output
will be saved alongside it (click the \emph{Preview} button or press
\emph{Cmd+Shift+K} to preview the HTML file).

The preview shows you a rendered HTML copy of the contents of the
editor. Consequently, unlike \emph{Knit}, \emph{Preview} does not run
any R code chunks. Instead, the output of the chunk when it was last run
in the editor is displayed.

\end{document}
